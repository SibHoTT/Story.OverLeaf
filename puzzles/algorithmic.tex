\section{Конструктивность или алгоритмичность}

\let\define\emph

Важно понимать, что термины <<конструктивность>> и \define{<<алгоритмичность>>} отнюдь не являются синонимами. \define{Конструктивность} объекта или процедуры несёт куда более широкий смысл, поскольку
не утверждает ограниченности его свойств только формализуемыми в классической теории алгоритмов в духе Чёрча, Тьюринга, Поста и др. Так, природа объекта не обязана соответствовать подразумеваемым
различными моделями ТА ограничениям типа фиксации исполнителя, дискретности, последовательности, детерминизма, отсутствия влияния <<внешних>> по отношению к системе <<исполнитель-программа>> факторов (взаимодействий, событий, исключений) и многого другого.
