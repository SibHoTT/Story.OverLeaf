\section{Вычислительное триединство}

\begin{figure}[ht] 
  \center
  %%\includegraphics [scale=0.27] {latex}
  \caption{Здесь следует поместить диаграмму в виде треугольника}
  %\caption{A nice triangle diagram should be put there } 
  \label{img:trinitary}  
\end{figure}

На современном этапе развития метаматематических подходов и исследования оснований математики 
очевидным стало единство между тремя ракурсами\footnote{дискурсами, подходами, frameworks, 
ТЗ\ldots} обзора или источниками языков, позволяющими унифицированно обращаться 
с разнообразием имеющихся в распоряжении человечества математических знаний:

\begin{itemize}
\item{ТД} - теория доказательств

\item{ТК} - теория категорий

\item{ТТ} - теория типов [Harper, видео+слайды Licata, Brunerie\ldots]
\end{itemize}

Это замечательное соответствие, получившее известность под названием \emph{computational 
trinitarianism}\footnote{Каков первичный источник?} (англ.)

\paragraph*{Вопросы, ассоциации, точки сборки и приклейки}
\begin{itemize}
\item Таблицы в [HoTT, RosS, $\cdots$]
\end{itemize}
