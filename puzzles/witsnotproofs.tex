\section{Свидетельства или доказательства}
Термы типа $T$, относимого к высказыванию $P$, традиционно называются его свидетельствами (\emph{witnesses}) [источники: везде, хоть HoTT]. Однако было бы ошибочным ассоциировать их с доказательствами\footnote{Термы-доказательства, термы-допущения (аксиомы...), ещё варианты? Интуиционизм.}.

Следует особо сконцентрироваться на том, что сопоставление определённому
высказыванию $P$ (в смысле некоторой логики, в частности, (супер-?)интуиционистской)
типа $T$ (в смысле теории типов, в частности, CCIC, sic!) само по себе не даёт 
оснований связывать [Browersdictum] какие-либо термы этого типа $t:T$, имеются они или нет, с
доказательствами $p$ высказывания $P$ в смысле теории доказательств или какими-либо
приводящими к $P$ последовательностями вывода. Например, аксиоматический характер 
высказывания $P$ может быть установлен принудительным введением (постулированием) 
терма a типа T. Стало быть, термы в PT-интерпретации могут представлять не только 
доказательства (тем более — необязательно различные), но и допущения, гипотезы... 
Область, из которой могут происходить иные возможности, затруднительно (было бы) 
даже определить. $\uparrow$ Впредь будем чаще всего разграничивать термы (в?) данного 
контекста $\textup{ctx}$ на допущения, допускаемые, допущенные (assumed) (а глобальные, термы 
сортов delta? Сжатие!!? Просто алиасы или нотации?) и доказанные (proven). Как 
именно подобное разделение осуществляется в Coq? ОРИСС! Найти подтверждающие 
написанное

\begin{itemize}
\item Как и чем именно <<в индустрии>> устанавливается соответствие между
формулировкой в ТТ и её версией на языках внутри Coq?
\end{itemize}